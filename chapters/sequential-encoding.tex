\subsection{Sequential scheme}


We continue with the sequential scheme, as presented in Section~\ref{sec:orderedScheme}.
In the parallel scheme, we simply translated every applicable rule into a clause, and asked the SAT-solver to find a model that satisfies all the clauses. 
Now, we model the state of the sentence by creating new variables at each rule application. 
To give an example, the sentence would look the following after applying the rule \t{REMOVE adj}:

\begin{equation}
\begin{array}{c c c c c}
\sobrePr & \unaPrn & \aproximacionN & \masAdv      & \cientifica{}'_\adj \\
\sobreN  & \unaDet &                & \mas{}'_\adj & \cientificaN \\
         & \unaPrsPThree \\
         & \unaPrsPOne \\
         & \unaImp \\
\end{array}
\end{equation}

In Section~\ref{sec:orderedScheme}, we explained how the version by \cite{lager_nivre01} is not a SAT-problem, but manipulation of Boolean expressions.
Then we established that we can make it into a SAT-problem by leaving the variables on the first round unassigned, which gives us the question ``which variables were originally true'', and promised that it will all make sense in Chapter~\ref{chapterCGana}. Regardless whether we use the scheme as a SAT-problem or not, in the remainder of this appendix, we will show how the new variables are created. 
We use the same rule types as we have seen for the parallel scheme.











%The first rule accesses the initial input by the morphological analyser, 
%and the $n^{th}$ rule accesses the sentence after the first $n-1$ rules have been run.
%Similarly to \cite{lager_nivre01}, we model this as a composition of clauses, each accessing the state produced by the previous clause. 
%Recall that all the variables start off as true, so we need no changes in the variables to be selected. The general procedure for rules with a single condition is explained in Section~\todo{write that section}; here we show the process for all rule types, as in the unordered scheme.

%each time that a rule is applied to a target, a new variable for the target is created.
%This makes it possible to separate two cases: a variable assigned a value, because some rule targets it, or a variable is assigned a value, because it appears in the condition of some other rule. Most importantly, it allows us to model the effect of time.




% The main difference is that the sentence starts off with all variables assigned $True$. Contrast \ref{:allStartTrue} with \ref{eq:atleastOneTrue}; 

% \begin{equation}
% \begin{array}{c}
% \sobrePr \wedge \sobreN \\
% \unaPrn \wedge \unaDet \wedge \unaPrsPThree \wedge \unaPrsPOne \wedge \unaImp \\
% \aproximacionN \\
% \masAdv \wedge \masAdj \\
% \cientificaAdj \wedge \cientificaN \\
% \end{array}
% \label{eq:allStartTrue}
% \end{equation}




% For subsequent rule applications, only the variables from the latest round are accessible: the $n^{th}$ rule may only access the variables that are created, or left unchanged, by the $n-1^{th}$ rule. 

% 

% \todo{Koen, you can stop reading here, this is still crap!}

% \begin{equation}
% \begin{array}{r @{~\eqdef~} l}
% \text{condsHold}     & c1 \vee c2 \vee c3 ... \\
% \text{someTrgIsTrue} & r1 \vee r2 \vee r3 ... \\
% \text{noOtherIsTrue} & \neg r4 \wedge \neg r5 ... \\
% \text{onlyTrgLeft}   & \text{someTrgIsTrue} \wedge \text{noOtherIsTrue} \\
% \text{cannotApply}   & \neg \text{condsHold} \vee \text{onlyTrgLeft} \\
% %
% \newVar trg' &  trg \wedge \text{cannotApply}
%               %  [ andl s newTrgName [ oldTrgLit     --wN<a> was also true, and
%               %                      , cannotApply ] --rule cannot apply 
%               %     | oldTrgLit <- trgPos
%               %     , let newTrgName = show oldTrgLit ++ "'" ]
% \end{array}
% \end{equation}

\subsubsection{No conditions} 


\cgrule{REMOVE adj} Unconditionally removes all readings which contain the target. We simply create a new variable with the value False.

\begin{equation}
\centering
  %\begin{array}{r @{~\eqdef~} l}
    % \mas{}'_\adj        & \text{New variable} \\
    % \cientifica{}'_\adj &  \text{New variable} \\
    % \multicolumn{2}{c}{\neg{}\mas{}'_\adj}  \\
    % \multicolumn{2}{c}{\neg{}\cientifica{}'_\adj}
%
\begin{array}{r @{~\eqdef~} l}
    \newVar \mas{}'_\adj         & \text{ \em False} \\
    \newVar \cientifica{}'_\adj  & \text{ \em False} \\
  \end{array}
\end{equation}

\cgrule{SELECT adj} Unconditionally removes all other readings that appear in the same cohort with the target. 
We treat ``{\sc select} target'' rules as ``{\sc remove} all but target'', and create new variables for everything but the target. 

\begin{equation}
\begin{array}{r @{~\eqdef~} l}
    \newVar \mas{}'_\adv      & \text{ \em False} \\
    \newVar \cientifica{}'_\n & \text{ \em False} \\
\end{array}
\label{eq:slAdjOrdered}
\end{equation}



\subsubsection{Positive conditions}

\def\invConds{\text{invalid condition}}
\def\onlyTrgLeft{\text{only target left}}

\begin{align}
%
\intertext{\cgrule{REMOVE adj IF (1 adj)} Single condition.} 
\newVar \mas{}'_\adj 
      & \eqdef \masAdj\ 
        \wedge\: ( \; \ob{\neg\cientificaAdj}^{\invConds} 
        \vee  \ob{\neg\masAdv}^{\onlyTrgLeft} ) \\
%
%
\intertext{\cgrule{REMOVE adj IF (1 adj) (-1 n)} Conjunction of conditions.}
\newVar \mas{}'_\adj 
      & \eqdef \masAdj\ 
       \wedge\: ( \; \ob{\neg ( \cientificaAdj \wedge \aproximacionN )\;}^{\invConds} 
       \vee  \ob{\neg\masAdv}^{\onlyTrgLeft} ) \\
%
%
\intertext{\cgrule{REMOVE adj IF (1C adj)} Careful context.} %The condition for \cgrule{1C adj} is $\cientificaAdj \wedge \neg \cientificaN$; we know by the default rule that one of them must be true, so just making $\cientificaN$ positive, we negate this
  \newVar \mas{}'_\adj 
     & \eqdef \masAdj\ 
     \wedge\: ( \; \ob{\neg ( \cientificaAdj \wedge \neg \cientificaN )\;}^{\invConds} 
     \vee \ob{\neg\masAdv}^{\onlyTrgLeft}  )  \\
%
%
\intertext{\cgrule{REMOVE adj IF (-1* n)} Scanning.}
  \newVar \mas{}'_\adj 
     & \eqdef \masAdj\ 
     \wedge\: ( \; \ob{\neg ( \aproximacionN \vee \sobreN )\;}^{\invConds} 
     \vee \ob{\neg\masAdv}^{\onlyTrgLeft} ) \\
%
%
\intertext{\cgrule{REMOVE adj IF (-1* n BARRIER det)} Scanning up to a barrier. }
  \newVar \mas{}'_\adj 
     & \eqdef \masAdj\ 
     \wedge\: ( \; \ob{\neg ( \aproximacionN \vee (\sobreN \wedge \neg\unaDet))\;}^{\invConds} 
     \vee \ob{\neg\masAdv}^{\onlyTrgLeft} )  \\
%
%
\intertext{\cgrule{REMOVE adj IF (-1* n CBARRIER det)} Scanning up to a careful barrier. }
  \newVar \mas{}'_\adj 
     & \eqdef \masAdj\ 
     \wedge\: ( \; \ob{\neg ( \aproximacionN \vee (\sobreN \wedge \unaAny)) }^{\invConds} 
     \vee \ob{\neg\masAdv}^{\onlyTrgLeft} ) \\
%
%
\intertext{\cgrule{REMOVE adj IF (-1C* n)} Scanning with careful context.} %The presence of $\sobrePr$, regardless of $\sobreN$, is enough to invalidate the C condition.}
  \newVar \mas{}'_\adj 
     & \eqdef \masAdj\ 
     \wedge\: ( \; \ob{\neg (\aproximacionN \vee (\sobreN \wedge \neg\sobrePr))\;}^{\invConds} 
     \vee \ob{\neg\masAdv}^{\onlyTrgLeft} ) \\
%
%
\intertext{\cgrule{REMOVE adj IF (-1C* n BARRIER det)} Scanning with careful context, up to a barrier.}
  \newVar \mas{}'_\adj 
     & \eqdef \masAdj\ 
     \wedge\: ( \; \ob{\neg (\aproximacionN \vee  (\sobreN \wedge \neg\sobrePr \wedge \neg\unaDet))\;}^{\invConds} 
     \vee \ob{\neg\masAdv}^{\onlyTrgLeft} )  \\
%
%
\intertext{\cgrule{REMOVE adj IF (-1C* n CBARRIER det)} Scanning with careful context, up to a careful barrier.}
  \newVar \mas{}'_\adj 
     & \eqdef \masAdj\ 
     \wedge\: ( \; \ob{\neg (\aproximacionN \vee  (\sobreN \wedge \neg\sobrePr \wedge \unaAny))\;}^{\invConds} 
     \vee \ob{\neg\masAdv}^{\onlyTrgLeft} ) 
\end{align}




\subsubsection{Inverted conditions}


\begin{align}
\intertext{\cgrule{REMOVE adj IF (NOT 1 adj)} Single inverted condition.}
  \newVar \mas{}'_\adj 
     & \eqdef \masAdj\ 
     \wedge\: ( \; \ob{\neg (\neg \cientificaAdj)\;}^{\invConds} 
     \vee \ob{\neg\masAdv}^{\onlyTrgLeft} ) \notag \\
     & \eqdef \masAdj\ 
     \wedge\: ( \; \cientificaAdj
     \vee  \neg\masAdv )  \\
%
\intertext{\cgrule{REMOVE adj IF (NOT 1 adj) (NOT -1 n)} Conjunction of inverted conditions.}
  \newVar \mas{}'_\adj 
     & \eqdef \masAdj\ 
     \wedge\: ( \; \ob{\neg (\neg \cientificaAdj \wedge \neg \aproximacionN)\;}^{\invConds} 
     \vee \ob{\neg\masAdv}^{\onlyTrgLeft} )  \\
%
%
\intertext{\cgrule{REMOVE adj IF (NEGATE -3 pr LINK 1 det LINK 1 n)} Inversion of a conjunction of conditions.}
%
  \newVar \mas{}'_\adj 
     & \eqdef \masAdj\ 
     \wedge\: ( \; \ob{\neg (\neg (\sobrePr \wedge \unaDet \wedge \aproximacionN))\;}^{\invConds} 
     \vee \ob{\neg\masAdv}^{\onlyTrgLeft} ) \notag \\
     & \eqdef \masAdj\ 
     \wedge\: ( \;  \sobrePr \wedge \unaDet \wedge \aproximacionN
     \vee \neg\masAdv )   \\
%
%
\intertext{\cgrule{REMOVE adj IF (NOT 1C adj)} Negated careful context. Condition cannot be unambiguously adjective.}
%
  \newVar \mas{}'_\adj 
     & \eqdef \masAdj\ 
     \wedge\: ( \; \ob{\neg (\neg (\cientificaAdj \wedge \neg\cientificaN ))\;}^{\invConds}
     \vee \ob{\neg\masAdv }^{\onlyTrgLeft} ) \\
%
%
\intertext{\cgrule{REMOVE adj IF (NOT -1* n)} Scanning. There must be no nouns before the target.}
%
  \newVar \mas{}'_\adj 
     & \eqdef \masAdj\ 
     \wedge\: ( \; \ob{\neg ( \sobreN \vee \aproximacionN )\;}^{\invConds}
     \vee \ob{\neg\masAdv }^{\onlyTrgLeft} ) \\
%
\intertext{\cgrule{REMOVE adj IF (NOT -1* n BARRIER det)} Scanning up to a barrier. There must be no nouns before the target, up to a determiner.}
%
  \newVar \mas{}'_\adj 
     & \eqdef \masAdj\ 
     \wedge\: ( \; \ob{\neg ( (\sobreN \vee \neg\unaDet) \vee \aproximacionN )\;}^{\invConds}
     \vee \ob{\neg\masAdv }^{\onlyTrgLeft} ) \\
%
%
\intertext{\cgrule{REMOVE adj IF (NOT -1* n CBARRIER det)} Scanning up to a careful barrier.}
%
  \newVar \mas{}'_\adj 
     & \eqdef \masAdj\ 
     \wedge\: ( \; \ob{\neg ( (\sobreN \vee \unaAny) \vee \aproximacionN )\;}^{\invConds}
     \vee \ob{\neg\masAdv }^{\onlyTrgLeft} ) 
%
%
%
\end{align}


% \todo{Difference between ``condition didn't hold in the first place'' and ``two rules have the same condition and conflicting action''}