\def\t#1{\texttt{#1}}


\chapter{Test Case Generation for Grammatical Framework}
\label{chapterGFtest}

Now we change to a completely new topic. 
Previously, we have enforced an internal logic to the grammar rules, without needing any interaction with the external world. We have taken for granted that the humans who write the rules know what they are doing; and even if this assumption is false, there are already methods that test whether a given rule takes effect in a corpus.


\section{The problem}

Traditionally, GF grammars are tested by the grammarians themselves,
much like unit testing. When implementing some feature, such as 
relative clauses, the grammarian comes up with a test suite of 
sentences that include relative clauses, and stores in the form of 
abstract syntax trees. In principle, a test suite created for one 
language can easily be reused for another, because the ASTs are 
identical. Ideally, every time someone touches relative clauses 
in the any concrete syntax, the trees in the test suite will be 
linearised with the changed concrete syntax, and verified by someone
who speaks the language (or compared to the original gold standard, 
if there is one). This scheme can fail for various reasons: 

\begin{itemize}
\item The original list is not exhaustive: for instance, it tests only
``X, who loves me'' but not ``X, whom I love''. 
\item The original list is exhaustive in one language, but not in all:
for instance, it started in English and only included one noun, but in
French it would need at least one masculine and one feminine noun. 
\item The list is overly long, with redundant test cases, and human
testers are not motivated to read through. 
\item A grammarian makes a change somewhere else in the grammar, and
does not realize that it affects relative clauses, and thus does not
rerun the appropriate test suite. 
\end{itemize}

\section{How it works}


\begin{figure}[h]

  \centering
    \begin{verbatim}
abstract Foods = {
  flags startcat = Comment ;
  cat
    Comment ; Item ; Kind ; Quality ;
  fun
    Pred : Item -> Quality -> Comment ;
    This, That, These, Those : Kind -> Item ;
    Mod : Quality -> Kind -> Kind ;
    Wine, Cheese, Fish, Pizza : Kind ;
    Very : Quality -> Quality ;
    Fresh, Warm, Good, Italian, 
      Expensive, Delicious, Vegan : Quality ;
}
    \end{verbatim}
  \caption{GF grammar}
\label{fig:exampleGrammar}
\end{figure}

Figure~\ref{fig:exampleGrammar} shows a small example of a GF grammar. We refer to this grammar throughout the section.

\paragraph{Test case} 
The basic unit of a test case is a single constructor. 
We start by building a set of trees using the constructor.
The constructor can be of any arity: if we are interested in a 0-place function, such as \t{Pizza}, then the subtree \t{Pizza} is the full set of trees. If we choose a function with arguments, such as \t{Pred}, then we do the following:
\begin{itemize}
\item For each argument type (\t{Item} and \t{Quality}), compute the set of minimal and representative trees. This is a recursive process: to compute the set of trees in \t{Item}, we must consider all functions that create its argument types (\t{Kind}), until we have a set of 0-place functions which to choose from.
\item Apply the constructor \t{Mod} to the combinations of the trees.
\end{itemize}

\paragraph{Testing the function \t{Very}}
Let us consider a Spanish concrete syntax, and test the function \t{Very : Quality -> Quality}.
\t{Very} takes a \t{Quality} and generates a \t{Quality}; for example, \t{Warm} to \t{Very Warm}.

Firstly, we need a minimal and representative set of arguments of type \t{Quality}, to \t{Very}. We can exclude trees that use the same constructor twice (i.e. trees that are already using \t{Very}), so we only concentrate on the set \t{Fresh, Warm, Good, Italian, Expensive, Delicious, Vegan}, and pick as many as are needed to highlight different grammatical phenomena in the particular concrete syntax.

The normal Spanish word order is noun--adjective, e.g. \emph{vino italiano} `Italian wine', but a few exceptional adjectives are placed in front of the noun, e.g. \emph{buen vino} `good wine'. Thus in order to cover the full spectrum of adjective placement, we need one premodifier and one postmodifier adjective. The two examples of the category \t{Quality} could be e.g. \t{\{Good, Vegan\}}. We finish the test cases by applying the constructor \t{Very} to them, which gives us the minimal and representative set \t{\{Very Good, Very Vegan\}}.


\paragraph{Context} 

Secondly, we create contexts for the previously chosen test cases. By \emph{context}, we mean simply a tree in some other category, with a \emph{hole} of type \t{Quality}.


\begin{figure}[h]

  \centering
  \todo{A picture of trees with holes}
 \caption{Two trees of type \t{Comment}, with a hole of type \t{Quality}}
\end{figure}

There are two ways that a \t{Quality} can end up in a \t{Comment}: by using \t{Pred : Item -> Quality -> Comment} or \t{Mod : Quality -> Kind -> Kind}. (It can also be used by another \t{Very}, but we exclude repetitions of the same constructor.) 

\paragraph{Context created by \t{Mod}} Firstly, let us look at the context created by \t{Mod : Kind -> Quality}.
We have our two examples of \t{Quality}, and we want to make sure that we see all the variation in them.
Spanish nouns have gender, hence a minimal and representative set of trees in category \t{Kind} could be e.g. \t{\{Pizza, Cheese\}}, where pizza is feminine and cheese is masculine. 

Applying the set of arguments \t{\{Very Good, Very Vegan\}} into the contexts \t{\{Mod ? Pizza, Mod ? Cheese\}}, we get the set \t{\{Mod (Very Good) Pizza, Mod (Very Good) Cheese, Mod (Very Vegan) Pizza, Mod (Very Vegan) Cheese\footnote{\url{http://realvegancheese.org/}}\}}. Note that \t{Kind} is unspecified for number, because it is still waiting for a determiner (\t{This, That, These} or \t{Those}) to complete it into \t{Item}. The type of \t{Kind} contains both singular and plural variants, and by linearising these 4 trees, we get the strings shown in Figure~\ref{fig:veganCheese}.

\begin{figure}
\centering
\begin{tabular}{| l | l |}
\hline
\t{Mod (Very Good) Pizza}   & \t{Mod (Very Good) Cheese} \\ 
muy buena pizza             & muy buen queso \\
muy buenas pizzas			& muy buenos quesos \\ \hline

\t{Mod (Very Vegan) Pizza}  & \t{Mod (Very Vegan) Cheese} \\
pizza muy vegana            & queso muy vegano \\
pizzas muy veganas          & quesos muy veganos \\ \hline
\end{tabular}
\caption{Variation of adjective forms and placement in category \t{Kind}}
\label{fig:veganCheese}
\end{figure}

\paragraph{Context created by \t{Pred}} 
Secondly, let us look at the context created by \t{Pred : Item -> Quality -> Comment}. Now the \t{Quality} is in a predicative position: ``X is very good/very vegan''.
We start by creating a representative and minimal set of arguments of type \t{Item}, in order to squeeze out all the variation from the test cases of type \t{Quality}.

\t{Item} is formed by using one of the four functions \t{This, That, These, Those : Kind -> Item}.
We remember from choosing \t{Kind}s that gender was important, and it is still relevant for \t{Item}: the predicative adjective should agree with the subject, so a pizza is \emph{vegana}, and cheese is \emph{vegano}. 
We have already found a set of \t{Kind}s to represent all genders, so we can reuse that set to create a representative set of \t{Item}s. But we need something more too---\t{Item} has an inherent number, which comes from the constructor function. \t{This} and \t{That} are singular, \t{These} and \t{Those} are plural, so we should choose one of each. Hence we end up with the following trees: \t{\{This Pizza, This Cheese, These Pizza, These Cheese\}}.

The presence or absence of additional adjectives wouldn't make any difference in Spanish: both \t{This Pizza} and \t{This (Very Italian) Pizza} result in choosing the string \emph{vegana} for the \t{Quality} in the predicative position. In the minimalist spirit, we prefer \t{This Pizza}: the resulting \t{Comment} is shorter, and it is easier for the oracle to judge the grammaticality of the predicative adjective, if there is only one adjective in the whole sentence.

\paragraph{Putting it all together}

piece the trees together

\paragraph{Shrinking}
We notice that only by testing one function, \t{Very}, we have actually quite a coverage of the grammar. 
Lot of the considerations are the same: number and grammatical gender play a role in a lot of stuff.
Easy step: don't show redundant trees to the user. One tree can actually test different phenomena.

\section{Technical details}

GF grammar compiles into a low-level format called PGF. After the
compilation, we get one category for each combination of parameters:
for English adjectives, \texttt{A => A$_{pos}$, A$_{comp}$,
A$_{superl}$}, and for Spanish, \texttt{A => A$_{pos×sg×masc}$, \dots,
A$_{superl×pl×fem}$}. 

Suddenly, we have a bunch of new types, and those are different for
each concrete syntax! The original question ``we need a sample of
nouns/verbs/… that makes sense'' can be simplified ``we need one
noun/verb/… of each type''. The types are determined by the parameters
in the concrete syntax. 

So remember all the hassle when you can't pattern match strings to
know something, but instead you have to define a parameter? This is
actually a nice side effect from that: each parameter contributes to a
new category, so it pays off in generating examples. If the feature is
important for your grammar---say that in language A, negation is
simply attaching the word  ``no'' before the verb, and in language B,
negation changes the word order and the object case. Then in the GF
grammar for language B, we would need a Boolean \texttt{isNeg} field
in the relevant categories, which we then pattern match against in
order to determine the relevant operations. That parameter in the
abstract category translates into different concrete categories, and
that way, when we generate example trees, we make sure to include one
of each. For instance, in language A, we could end up with the trees
``any horse'' and ``all horses'' when testing NPs, but in language B,
the set would also include ``no horses''. 



\section{Evaluation}


\begin{itemize}
\item Cost
  \begin{itemize}
  \item time of generating examples
  \item time of looking at examples
  \end{itemize}

\item Effect
  \begin{itemize}
  \item compare against other methods -- what methods?
  \item For application grammars, if you're writing them from scratch, it is actually pretty feasible to just gt the hell out of it as you write. But this doesn't work for bigger grammars.
  \item Morphology can be tested efficiently againts any existing morphological analyser. I've used Apertium for Dutch and Basque.
  \end{itemize}
\end{itemize}


\section{Future work}

We plan to look into existing text corpora, and find trees that are
structurally identical  to those that our program generates as a
minimal and representative example. As a simplified example, ``a worm
without winter'', generated by the program, would be identical\footnote{This particular example holds for English, but in another language, the words ``pizza'' and ``worm'', as well as ``winter'' and ``cheese'' may not match in all relevant features---grammatical gender, whether the word starts with a vowel or a consonant, etc. All this information comes from the concrete syntax!} 
in structure to ``a pizza without cheese'', found in a real text, and
can thus be substituted for the generated one.   
Alternatively, we could use statistical information on co-occurrences
of words, and generate appropriate pools of words, from which we draw
example sentences. 
