\def\t#1{\texttt{#1}}

\chapter{Background}
\label{chapterBackground}

\epigraph{\it Much of the methodology of grammar testing is dictated by common sense.}{Miriam Butt et al.\\{\it A Grammar Writer's Cookbook}}

In this chapter, we present the main areas of this thesis: computational natural language grammars and software testing.
Section~\ref{sec:cg-intro} introduces the CG formalism and some of the different variants, along with the historical context and related work.
For a description of any specific CG implementation, we direct the reader to the original source: CG-1 \cite{karlsson1990cgp,karlsson1995constraint}, CG-2 \cite{tapanainen1996}, VISL CG-3 \cite{bick2015,vislcg3}.
Section~\ref{sec:gf-intro} introduces the GF formalism \cite{ranta2011gfbook} and the Resource Grammar Library \cite{ranta2009rgl}.

Section~\ref{sec:testing-intro} is a brief and high-level introduction to software testing and verification, aimed for a reader with no prior knowledge on the topic. For a more thorough introduction on software testing in general, we recommend \citet{ammann2016introduction}, and for SAT in particular, we recommend \citet{biere2009handbook}.



% \todo{quote from a LFG book---maybe find some use for it}
% \begin{quote}
% Testing the grammar is an indispensable part of ensuring robustness
% and increasing the performance of a grammar. Although individual constructions
% can be tested with a few sample sentences as they are developed,
% more systematic testing is required to maintain the quality of
% the grammar over time and to catch unexpected results of the addition
% of new rules and lexical items. Much of the methodology of grammar
% testing is dictated by common sense. However, issues do arise with respect
% to what kinds of testsuites to use, what other kinds of methods
% apart from testsuites one might use to test the grammar for continued
% consistency, and how to get a stable measure of the grammar's performance
% as it grows and changes.
% \end{quote}

% \def\pmcfg{\textsc{pmcfg}}
% \def\gf{\textsc{gf}}
% \def\cg{\textsc{cg}}
% \def\fsig{\textsc{fsig}}
\def\pmcfg{PMCFG}
\def\gf{GF}
\def\cg{CG}
\def\fsig{FSIG}
\newcommand{\quality}[1]{${\tt Quality_{#1}}$}
\newcommand{\kind}[1]{${\tt Kind_{#1}}$}
\newcommand{\very}[1]{${\tt Very_{#1}}$}
\newcommand{\comment}{${\tt Comment}$}
\newcommand{\modFun}[2]{${\tt Mod_{#1\times#2}}$}
\newcommand{\predFun}[3]{${\tt Pred_{#1\times#2\times#3}}$}
\newcommand{\itemSpa}[2]{${\tt Item_{#1\times#2}}$}
\newcommand{\itemEng}[1]{${\tt Item_{#1}}$}

\section{Constraint Grammar}
\label{sec:cg-intro}

Constraint Grammar (\cg{}) is a formalism for 
disambiguating morphologically analyzed text. 
It was first introduced by Fred Karlsson  
\cite{karlsson1990cgp,karlsson1995constraint}, and has been used for
many tasks in computational linguistics, such as part-of-speech
tagging, surface syntax and machine translation \cite{bick2011}.
\cg{}-based taggers are reported to achieve F-scores of over 99 \% for
morphological disambiguation, and around 95-97 \% for syntactic analysis
\cite{bick2000palavras,bick2003hybridCG_PSG,bick2006spanish}. 
\cg{} disambiguates morphologically analyzed input by using constraint
rules which can select or remove a potential analysis (called
\emph{reading}) for a target word, depending on the context words
around it.  Together these rules disambiguate the whole text.


In the example below, we show an initially ambiguous sentence ``the bear
sleeps''. 
It contains three word forms, such as \t{"<bear>"}, each followed by its \emph{readings}.
A reading contains one lemma, such as \t{"bear"}, and a list of morphological tags, such as \t{noun sg}.
%Additional lemmas within one word form, such as clitic pronouns, are represented as \emph{subreadings}; each indented one more tab.
A word form together with its readings is called a \emph{cohort}. A cohort is ambiguous, if it contains more than one reading.

\begin{figure}[h]
\centering
\ttfamily
\begin{tabular}{p{0.6cm} l  p{0.6cm} l}
"<the>"  &                & "<sleeps>"        \\
    & "the" det def       &     & "sleep" noun pl \\
"<bear>" &                &     & "sleep" verb pres p3 sg \\
    & "bear" noun sg      & "<.>"                   \\
    & "bear" verb pres    &     & "." sent          \\
    & "bear" verb inf \\
\end{tabular}
\label{fig:theBearSleeps}
\caption{Ambiguous sentence {\em the bear sleeps.}}
\end{figure}


\noindent We can disambiguate this sentence with two rules:

\begin{enumerate}
\def\labelenumi{\arabic{enumi}.}
\itemsep1pt\parskip0pt\parsep0pt
\item \texttt{REMOVE verb IF (-1 det)}
  `Remove verb after determiner'
\item  \texttt{REMOVE noun IF (-1 noun)}
  `Remove noun after noun'
\end{enumerate}

\noindent Rule 1 matches the word \emph{bear}: it is tagged as verb and is
preceded by a determiner. The rule removes both verb readings from
\emph{bear}, leaving it with an unambiguous analysis \texttt{noun sg}.
Rule 2 is applied to the word \emph{sleeps}, and it removes the noun
reading. The finished analysis is shown below:

\begin{itemize}
\item[] 
\begin{verbatim}
"<the>"
        "the" det def
"<bear>"
        "bear" noun sg
"<sleeps>"
        "sleep" verb pres p3 sg
\end{verbatim}
\end{itemize}

It is also possible to add syntactic tags and dependency structure within \cg{} \cite{vislcg3,bick2015}.
However, for the remainder of this introduction, we will illustrate the examples with the 
most basic operations, that is, disambiguating morphological tags.
The syntactic disambiguation and dependency features are not fundamentally
different from morphological disambiguation: the rules describe an \emph{operation}
performed on a \emph{target}, conditional on a \emph{context}.

\subsection{Related work}

\cg{} is one in the family of shallow and reductionist grammar formalisms. Disambiguation using constraint rules dates back to 1960s and 1970s---the closest system to modern \cg{} was Taggit \cite{taggit}, which was used to tag the Brown Corpus.
Karlsson \cite{karlsson1995constraint} lists various approaches to the disambiguation 
problem, including manual intervention, statistical optimisation, unification and Lambek calculus. 
For disambiguation rules based on local constraints, Karlsson mentions \cite{hindle1989disamrules,herz1991local}.

\cg{} itself was introduced in 1990. Around the same time, a related formalism was proposed: 
finite-state parsing and disambiguation system using constraint rules \cite{koskenniemi90}, which was later named Finite-State Intersection Grammar (\fsig{}) \cite{piitulainen1995}. 
Like \cg{}, a \fsig{} grammar contains a set of rules which remove impossible readings based on contextual tests, 
in a parallel manner: a sentence must satisfy all individual rules in a given \fsig{}. 
Due to these similarities, the name Parallel Constraint Grammar was also suggested \cite{koskenniemi97}.
% However, \fsig{} rules usually target syntactic phenomena, and the rules allow for more expressive contextual tests than \cg{}. 
% In this thesis, we use the name \fsig{} to refer to the framework that is aimed at producing a full syntactic analysis with the more expressive rules, 
% and PCG to describe any \cg{}-variant which happens to be parallel.
% Thus, we will call implementations such as \cite{voutilainen1994designing} an \emph{\fsig{} grammar} and a program that parses it an \emph{\fsig{} parser}.
% Conversely, the \cg{} parser in \cite{listenmaa_claessen2015} 
% and the ``\cg{}-like'' system in \cite{lager98} are instances of PCG.
%We return to the comparison with \fsig{} in Sections~\ref{sec:ordering}~and~\ref{sec:expressivity}.

Brill tagging \cite{brill1995} is based on transformation rules: the
starting point of an analysis is just one tag, the most common one,
and subsequent rule applications transform one tag into another, based
on contextual tests. Like \cg{}, Brill tagging is known to be efficient
and accurate. The contextual tests are very similar: Lager
\cite{lager01transformation} has automatically learned both Brill rules and \cg{} rules, using the same system.

Similar ideas to \cg{} have been also explored in other frameworks, such as finite-state automata \cite{gross1997local,grana2003fst},
%, such as Local grammars \cite{gross1997local} and finite-state contextual rules \cite{grana2003fst},
logic programming \cite{oflazer97votingconstraints,lager98}, 
constraint satisfaction \cite{padro1996csp}, 
and dependency syntax \cite{tapanainen97fdg}. 
 In addition, there are a number of reimplementations of \cg{} using finite-state methods \cite{yli-jyra2011cg_engine,hulden2011cg_engine,peltonen2011}. 

\subsection{Properties of Constraint Grammar}\label{sec:properties}

Karlsson \cite{karlsson1995constraint} lists 24 design principles and describes
related work at the time of writing.
Here we summarize a set of main features, and relate \cg{} to the developments in grammar formalism since the initial description.

\cg{} is a \emph{reductionistic} system: the analysis starts from a list of alternatives,
and removes those which are impossible or improbable.
\cg{} is designed primarily for analysis, not generation; its task is 
to give correct analyses to the words in given sentences,
not to describe a language as a collection of ``all and only the grammatical sentences''.

The syntax is decidedly \emph{shallow}: the rules do not aim to
describe all aspects of an abstract phenomenon such as noun phrase; 
rather, each rule describes bits and pieces with concrete conditions.
The rules are self-contained and mutually independent---this makes it 
easy to add exceptions, and exceptions to exceptions, without 
changing the more general rules.
% The rules are self-contained and independent.
% On the one hand, this provides no guarantee that a grammar is internally consistent.
% On the other hand, these features provide flexibility that is hard to mimic by a deeper formalism.
% As we have seen in the previous sections, rules can target individual words
% or other properties that are not generalisable to a whole word class,
% such as verbs that express cognitive processes.
% Introducing a subset of verbs, even if they are used only in one rule,
% is very cheap and does not create a complicated taxonomy of different verb types.

% Most importantly, the independence of rules makes \cg{} highly robust.
% If one of the words is unknown or misspelt, a generative grammar would fail to produce any analysis. 
% \cg{} would, at worst, just leave that part ambiguous, and do as good a job it can elsewhere in the sentence.


There are different takes on how \emph{deterministic} the rules are.
The current state-of-the-art \cg{} parser VISL CG-3 executes the rules strictly 
based on the order of appearance, but there are other implementations which 
apply their own heuristics, or remove the ordering completely, 
applying the rules in parallel. 
%---furthermore, there are several implementations of \cg{} with different kind of application orders,
%such as ``apply in the order introduced in the grammar file'' or ``apply in order of longer matching context''.
A particular rule set may be written with one application order in mind, but another party may 
run the grammar with another implementation---if there are any conflicting rule pairs, then the behaviour of the grammar is different.
For that reason, we decided to apply software testing and verification methods to \cg{} grammars.





\section{Grammatical Framework}
\label{sec:gf-intro}

Grammatical Framework (GF) \cite{ranta2011gfbook} 
is a framework for building multilingual grammar applications. Its main
components are a functional programming language for writing grammars
and a resource library that contains the linguistic details of many
natural languages.
A GF program consists of an \emph{abstract syntax} (a set of functions
and their categories) and a set of one or more
\emph{concrete syntaxes} which describe how the abstract
functions and categories are linearized (turned into surface strings) in each
respective concrete language. The resulting grammar
describes a mapping between concrete language strings and
their corresponding abstract trees (structures of function names).
This mapping is bidirectional---strings can be \emph{parsed} to
trees, and trees \emph{linearized} to strings.
As an abstract syntax can have multiple corresponding concrete syntaxes,
the respective languages can be automatically \emph{translated} from one to the other by
first parsing a string into a tree and then linearizing the obtained tree
into a new string.

\subsection{Related work}

\gf{} comes from the theoretical background of type theory and logical
frameworks, such as Montague grammar \cite{montague} and
Categorial grammar \cite{catgram}.


% Montague, R. Formal Philosophy. Collected papers edited by Richmond
% Thomason. New Haven: Yale University Press, 1974. The pioneering
% formalization of syntax and semantics together; GF can be seen as a
% framework for Montague-style grammars.


Abstract and concrete syntax: compiler construction \& tectogrammatical/phenogrammatical \todo{cite (Curry 1961)}. Multi-source multi-target compiler is multilingual by design, but tecto/pheno stuff wasn't multilingual before GF.


Ranta \cite{ranta2011gfbook} mentions a few grammar formalisms that
also build upon abstract and concrete syntax
\cite{deGroote2001acg,pollard2004hog,muskens2001lambda}. However, none
of these systems have focused on multilinguality.







\begin{figure}[h]
\centering

\begin{Shaded}
\begin{Highlighting}[]
\NormalTok{abstract }\DataTypeTok{Foods} \FunctionTok{=} \NormalTok{\{}
  \NormalTok{flags startcat }\FunctionTok{=} \DataTypeTok{Comment} \NormalTok{;}
  \NormalTok{cat}
    \DataTypeTok{Comment} \NormalTok{; }\DataTypeTok{Item} \NormalTok{; }\DataTypeTok{Kind} \NormalTok{; }\DataTypeTok{Quality} \NormalTok{;}
  \NormalTok{fun}
    \DataTypeTok{Pred}\OtherTok{ :} \DataTypeTok{Item} \OtherTok{->} \DataTypeTok{Quality} \OtherTok{->} \DataTypeTok{Comment} \NormalTok{;                 }\CommentTok{-- this wine is good}
    \DataTypeTok{This}\NormalTok{, }\DataTypeTok{That}\NormalTok{, }\DataTypeTok{These}\NormalTok{, }\DataTypeTok{Those}\OtherTok{ :} \DataTypeTok{Kind} \OtherTok{->} \DataTypeTok{Item} \NormalTok{;           }\CommentTok{-- this wine}
    \DataTypeTok{Mod}\OtherTok{ :} \DataTypeTok{Quality} \OtherTok{->} \DataTypeTok{Kind} \OtherTok{->} \DataTypeTok{Kind} \NormalTok{;                     }\CommentTok{-- Italian wine}
    \DataTypeTok{Wine}\NormalTok{, }\DataTypeTok{Cheese}\NormalTok{, }\DataTypeTok{Fish}\NormalTok{, }\DataTypeTok{Pizza}\OtherTok{ :} \DataTypeTok{Kind} \NormalTok{;}
    \DataTypeTok{Warm}\NormalTok{, }\DataTypeTok{Good}\NormalTok{, }\DataTypeTok{Italian}\NormalTok{, }\DataTypeTok{Vegan}\OtherTok{ :} \DataTypeTok{Quality} \NormalTok{;}
\end{Highlighting}
\end{Shaded}
  \caption{Abstract syntax of a GF grammar about food}
\label{fig:foods}
\end{figure}

\subsection{Abstract syntax}


Abstract syntax describes the constructions in our grammar without
giving a concrete implementation. Figure~\ref{fig:foods}
shows the abstract syntax of a small example grammar in GF, slightly
modified from \cite{ranta2011gfbook}, and Figure~\ref{fig:spanish}
shows a corresponding Spanish concrete syntax. We refer to this
grammar throughout the thesis. 

Section~\t{cat} introduces the categories of the grammar: \t{Comment},
\t{Item},  \t{Quality}, and \t{Kind}.  \t{Comment} is the \emph{start
  category} of the grammar: this means that only comments are complete
constructions in the language, everything else is an intermediate
stage. \t{Quality} describes properties of foods, such as
\t{Warm} and \t{Good}. %, and it can be used both for modification and predication.
\t{Kind} is a basic type for foodstuffs such as \t{Wine} and
\t{Pizza}: we know what it is made of, but everything else is
unspecified. In contrast, an \t{Item} is \emph{quantified}: we know if
it is singular or plural (e.g. `one pizza' vs. `two pizzas'), definite or
indefinite (`the pizza' vs. `a pizza'), and other such things (`your
pizza' vs. `my pizza'). 

Section~\t{fun} introduces functions: they are either lexical
items without arguments, or syntactic functions which manipulate their
arguments and build new terms. Of the syntactic functions, \t{Pred}
constructs an \t{Comment} from an \t{Item} and a \t{Quality},
building trees such as \t{Pred~(This~Pizza)~Good} `this pizza is
good'. 
\t{Mod}~adds an \t{Quality} to a \t{Kind}, e.g. \t{Mod~Italian~Pizza}
`Italian pizza'. The functions  \t{This, That, These} 
and \t{Those} quantify a \t{Kind} into an \t{Item}, for instance,
\t{That~(Mod~Italian~Pizza)} `that Italian pizza'.

\begin{figure}[h]
\centering
\begin{Shaded}
\begin{Highlighting}[]
\NormalTok{concrete }\DataTypeTok{FoodsSpa} \KeywordTok{of} \DataTypeTok{Foods} \FunctionTok{=} \NormalTok{\{}
  \NormalTok{lincat}
    \DataTypeTok{Comment} \FunctionTok{=} \DataTypeTok{Str} \NormalTok{;}
    \DataTypeTok{Item} \FunctionTok{=} \NormalTok{\{ s }\OtherTok{:} \DataTypeTok{Str} \NormalTok{; n }\OtherTok{:} \DataTypeTok{Number} \NormalTok{; g }\OtherTok{:} \DataTypeTok{Gender} \NormalTok{\} ;}
    \DataTypeTok{Kind} \FunctionTok{=} \NormalTok{\{ s }\OtherTok{:} \DataTypeTok{Number} \OtherTok{=>} \DataTypeTok{Str} \NormalTok{; g }\OtherTok{:} \DataTypeTok{Gender} \NormalTok{\} ;}
    \DataTypeTok{Quality} \FunctionTok{=} \NormalTok{\{ s }\OtherTok{:} \DataTypeTok{Number} \OtherTok{=>} \DataTypeTok{Gender} \OtherTok{=>} \DataTypeTok{Str} \NormalTok{; p }\OtherTok{:} \DataTypeTok{Position} \NormalTok{\} ;}
  \NormalTok{lin}
    \DataTypeTok{Pred} \NormalTok{np ap }\FunctionTok{=} \NormalTok{np}\FunctionTok{.}\NormalTok{s }\FunctionTok{++} \NormalTok{copula }\FunctionTok{!} \NormalTok{np}\FunctionTok{.}\NormalTok{n }\FunctionTok{++} \NormalTok{ap.s }\FunctionTok{!} \NormalTok{np}\FunctionTok{.}\NormalTok{n }\FunctionTok{!} \NormalTok{np}\FunctionTok{.}\NormalTok{g ;}
    \DataTypeTok{This}\NormalTok{ cn} \FunctionTok{=} \NormalTok{mkItem }\DataTypeTok{Sg} \StringTok{"este"} \StringTok{"esta"} \NormalTok{cn ;}
    \DataTypeTok{These}\NormalTok{ cn} \FunctionTok{=} \NormalTok{mkItem }\DataTypeTok{Pl} \StringTok{"estos"} \StringTok{"estas"} \NormalTok{cn ;}
    \CommentTok{-- That, Those defined similarly}
    \DataTypeTok{Mod} \NormalTok{ap cn }\FunctionTok{=} \NormalTok{\{ s }\FunctionTok{=} \NormalTok{\textbackslash\textbackslash{}n }\OtherTok{=>} \NormalTok{preOrPost ap}\FunctionTok{.}\NormalTok{p (ap}\FunctionTok{.}\NormalTok{s }\FunctionTok{!} \NormalTok{n }\FunctionTok{!} \NormalTok{cn}\FunctionTok{.}\NormalTok{g) (cn}\FunctionTok{.}\NormalTok{s }\FunctionTok{!} \NormalTok{n) ;}
                  \NormalTok{g }\FunctionTok{=} \NormalTok{cn}\FunctionTok{.}\NormalTok{g \} ;}
    \CommentTok{--Wine, Cheese, \dots, Italian, Vegan defined as lexical items}
  \NormalTok{param}
    \DataTypeTok{Number} \FunctionTok{=} \DataTypeTok{Sg} \FunctionTok{|} \DataTypeTok{Pl} \NormalTok{;}
    \DataTypeTok{Gender} \FunctionTok{=} \DataTypeTok{Masc} \FunctionTok{|} \DataTypeTok{Fem} \NormalTok{;}
    \DataTypeTok{Position} \FunctionTok{=} \DataTypeTok{Pre} \FunctionTok{|} \DataTypeTok{Post} \NormalTok{;}
  \NormalTok{oper}
\NormalTok{    mkItem} \NormalTok{num mascDet femDet cn} \FunctionTok{=}
     \KeywordTok{let} \NormalTok{det }\FunctionTok{=} \KeywordTok{case} \NormalTok{cn}\FunctionTok{.}\NormalTok{g }\KeywordTok{of} \NormalTok{\{ }\DataTypeTok{Masc} \OtherTok{=>} \NormalTok{mascDet ; }\DataTypeTok{Fem} \OtherTok{=>} \NormalTok{femDet \} ;}
      \KeywordTok{in} \NormalTok{\{ s }\FunctionTok{=} \NormalTok{det }\FunctionTok{++} \NormalTok{cn}\FunctionTok{.}\NormalTok{s }\FunctionTok{!} \NormalTok{num} \NormalTok{;} \NormalTok{n }\FunctionTok{=} \NormalTok{num} \NormalTok{; g }\FunctionTok{=} \NormalTok{cn}\FunctionTok{.}\NormalTok{g \} ;}
    \NormalTok{copula }\FunctionTok{=} \NormalTok{table \{ }\DataTypeTok{Sg} \OtherTok{=>} \StringTok{"es"} \NormalTok{; }\DataTypeTok{Pl} \OtherTok{=>} \StringTok{"son"} \NormalTok{\} ;}
    \NormalTok{preOrPost p x y }\FunctionTok{=} \KeywordTok{case} \NormalTok{p }\KeywordTok{of} \NormalTok{\{ }\DataTypeTok{Pre} \OtherTok{=>} \NormalTok{x }\FunctionTok{++} \NormalTok{y ; }\DataTypeTok{Post} \OtherTok{=>} \NormalTok{y }\FunctionTok{++} \NormalTok{x \} ;}
\end{Highlighting}
\end{Shaded}
  \caption{Spanish concrete syntax of a GF grammar about food}
\label{fig:spanish}
\end{figure}

\subsection{Concrete syntax}
\label{concrete_spanish_foods}
Concrete syntax is an implementation of the abstract syntax.
The section~\t{lincat} corresponds to \t{cat} in the abstract syntax:
for every abstract category introduced in \t{cat}, we give a concrete
implementation in \t{lincat}.


Figure~\ref{fig:spanish} shows the Spanish concrete
syntax, in which \t{Comment} is a string, and the rest of the
categories are more complex records. For instance, \t{Kind} has a
field \t{s} which is a table from number to string (\textsc{sg} $\Rightarrow$
\emph{pizza}, \textsc{pl} $\Rightarrow$ \emph{pizzas}), and another
field \t{g}, which contains its gender (feminine for \t{Pizza}). We
say that \t{Kind} has \emph{inherent} gender, and \emph{variable} number. 

The section~\t{lin} contains the concrete implementation of the
functions, introduced in \t{fun}. Here we handle
language-specific details such as agreement: when \t{Pred (This Pizza)
  Good} is linearized in Spanish, `esta pizza es buena', the copula must be singular
(\emph{es} instead of plural \emph{son}), and the adjective must be in singular
feminine (\emph{buena} instead of masculine \emph{bueno} or plural
\emph{buenas}), matching the gender of \t{Pizza} and the number of \t{This}. 
If we write an English concrete syntax, then only the number of the copula is
relevant: this pizza/wine \emph{is} good, these pizzas/wines \emph{are} good.


\subsection{PMCFG}
\label{sec:PMCFG}

\gf{} grammars are compiled into parallel multiple context-free
grammars (\pmcfg). Here we explain three key features, which will be
used for the test suite generation.

\paragraph{Concrete categories}

For each category in the original grammar, the \gf{} compiler
introduces a new \emph{concrete category} in the \pmcfg{} for each combination of
inherent parameters.  
These concrete categories can be linearized to strings or vectors of
strings. The start category (\t{Comment} in the Foods grammar) is in
general a single string, but intermediate categories may have to keep
several options open. 

Consider the categories \t{Item}, \t{Kind} and \t{Quality} in the
Spanish concrete syntax. Firstly, \t{Item} has inherent number
and gender, so it compiles into four concrete categories:
\itemSpa{sg}{masc}, \itemSpa{sg}{fem}, \itemSpa{pl}{masc} and
\itemSpa{pl}{fem}, each of them containing one string. Secondly,
\t{Kind} has an inherent gender and variable number, so it compiles into
two concrete categories: \kind{masc} and \kind{fem}, each of them a
vector of two strings (singular and plural). Finally, \t{Quality} needs to
agree in number and gender with its head, but it has its position as
an inherent feature.  Thus \t{Quality} compiles into two concrete
categories: \quality{pre} and \quality{post}, each of them a vector of
four strings.
% ({\stackanchor{\tt \small pl}{\tt \small sg}}
%  $\times$ {\stackanchor{\tt \small masc}{\tt \small fem}}).

\paragraph{Concrete functions}
Just like categories, each syntactic function from the original
grammar turns into multiple syntactic functions into the
\pmcfg{}---one for each combination of parameters of its arguments.

\begin{itemize}
\setlength\itemsep{0em}
\item[--] \modFun{pre}{fem~~} \t{:} \quality{pre~} $\rightarrow$ \kind{fem~} $\rightarrow$ \kind{fem}
\item[--]  \modFun{post}{fem~} \t{:} \quality{post} $\rightarrow$ \kind{fem~} $\rightarrow$ \kind{fem}
\item[--]  \modFun{pre}{masc~~}\t{:} \quality{pre~} $\rightarrow$ \kind{masc} $\rightarrow$ \kind{masc}
\item[--] \modFun{post}{masc} \t{:} \quality{post} $\rightarrow$ \kind{masc} $\rightarrow$ \kind{masc}
\end{itemize}


\paragraph{Coercions}
\label{sec:Coercions}
As we have seen, \t{Quality} in Spanish compiles into \quality{pre} and
\quality{post}. However, the difference of position is meaningful only when the
adjective is modifying the noun: ``la \emph{buena} pizza'' vs. ``la pizza
\emph{vegana}''. But when we use an adjective in a predicative position, both
classes of adjectives behave the same: ``la pizza es \emph{buena}''
and ``la pizza es \emph{vegana}''. As an optimization strategy, the
grammar creates a {\it coercion}: both \quality{pre} and \quality{post}
may be treated as \quality{*} when the distinction doesn't matter. 
Furthermore, the function \t{Pred : Item $\rightarrow$ Quality $\rightarrow$ S} uses
the coerced category \quality{*} as its second argument, and thus
expands only into 4 variants, despite there being 8 combinations of
\t{Item}$\times$\t{Quality}.

\begin{itemize}
\setlength\itemsep{0em}
\item[--] \predFun{sg}{fem}{*~} \t{:} \itemSpa{sg}{fem~} $\rightarrow$ \quality{*} $\rightarrow$ \comment
\item[--]  \predFun{pl}{fem}{*~} \t{:} \itemSpa{pl}{fem~} $\rightarrow$ \quality{*} $\rightarrow$ \comment
\item[--]  \predFun{sg}{masc}{*} \t{:} \itemSpa{sg}{masc} $\rightarrow$ \quality{*} $\rightarrow$ \comment
\item[--] \predFun{pl}{masc}{*} \t{:} \itemSpa{pl}{masc} $\rightarrow$ \quality{*} $\rightarrow$ \comment
\end{itemize}



% \subsection{Abstract syntax}

% Abstract syntax is a description of the things we can say in the
% language: it doesn't tell how, but just
% what. Figure~\ref{fig:abstract_syntax} shows a small example grammar
% for greetings.
% %: ``Hello world'' and ``Hello friends''.


% \begin{figure}[h]
%   \centering
% \begin{verbatim}
% abstract Hello = {
%   flags startcat = Greeting ;

%   cat 
%     Greeting ; Recipient ;

%   fun
%     Hello : Recipient -> Greeting ;
%     World : Recipient ;
%     Friends : Recipient ; 
% }

% \end{verbatim}
% \caption{Abstract syntax of a small GF grammar}
% \label{fig:abstract_syntax}
% \end{figure}

% Section \t{cat} introduces the categories of the grammar:
% \t{Greeting} and \t{Recipient}. 
% Of the two categories, \t{Greeting} is the \emph{start category}.
% This means that greetings are complete constructions in the language,
% everything else (in this grammar, only \t{Recipient} is an
% intermediate stage. However, it is perfectly fine to inspect any
% category, and generate or parse examples in them. (As we will see later
% in Chapter~\ref{chapterGFtest}, it makes sense to test grammars in
% smaller pieces.)

% The next section \t{fun} introduces functions: they are either lexical
% items without arguments, or syntactic functions which manipulate their
% arguments and build new terms. In this grammar, we have two lexical
% items, \t{World} and \t{Friends}, both of type \t{Recipient}, and one
% syntactic function, \t{Hello}, which takes a \t{Recipient} as an
% argument and produces a \t{Greeting}. 

% This grammar is very simple: the exhaustive list of abstract syntax
% trees  is just \t{\{Hello World, Hello Friends\}}.


% \subsection{Concrete syntax}

% Concrete syntax is the implementation of the abstract syntax.

% \begin{figure}[h]
%   \centering
% \begin{verbatim}
% concrete HelloEng of Hello = {

%  lincat 
%    Greeting, Recipient = Str ;

%  lin
%    Hello rec = "hello" ++ rec ;
%    World     = "world" ;
%    Friends   = "friends" ;
% }
% \end{verbatim}
% \caption{English concrete syntax of the GF grammar in Figure~\ref{fig:abstract_syntax}}
% \label{fig:concrete_syntax_eng}
% \end{figure}

% The section \t{lincat} contains the concrete types of the categories. 
% It corresponds to \t{cat} in the abstract syntax: for every abstract category
% introduced in \t{cat}, we give a concrete implementation in \t{lincat}.
% Figure~\ref{fig:concrete_syntax_eng} shows the English concrete
% syntax, in which both \t{Greeting} and \t{Recipient} are just strings.

% The section \t{lin} contains the concrete implementation of the
% functions, introduced in \t{fun}. The lexical items \t{World} and
% \t{Friends} are just strings, and \t{Hello} simply prefixes the word
% ``hello'' to its argument, resulting in ``hello world'' and ``hello friends''.

% However, the concrete syntax types can be more complex than just
% strings. GF compiles into a formalism called \pmcfg{}
% \cite{seki91pmcfg}, which is beyond context-free.

% \begin{figure}[h]
%   \centering
% \begin{verbatim}
% concrete HelloIce of Hello = {

%  lincat 
%    Greeting  = Str ;
%    Recipient = { s : Str ; n : Number } ;

%  lin
%    Hello rec = case rec.n of {
%                  Sg => "sæll" ;
%                  Pl => "sælir" } ++ rec ;
%    World     = {s = "heimur" ; n = Sg } ;
%    Friends   = {s = "vinir"  ; n = Pl } ;
% }
% \end{verbatim}
% \caption{Icelandic concrete syntax of the GF grammar in Figure~\ref{fig:abstract_syntax}}
% \label{fig:concrete_syntax_ice}
% \end{figure}

\section{Software testing and verification}
\label{sec:testing-intro} 

\begin{itemize}
\item All software has bugs.
\item Grammars are software.
\end{itemize}

\noindent The intelligent reader may complete the syllogism.

We can approach the elimination of bugs in two ways: reveal them
by constructing tests, or build safeguards into the program that
make it more difficult for bugs to occur in the first place. The
latter approach is more of a concern for developers of new programming
languages---the language can impose more checks and constraints on the
programmer, and thus make it harder to write buggy code. However, we
approach the problem from particular, existing grammar formalisms that
already have millions of lines of code written in them. Thus, we want
to develop methods that help finding bugs in existing software.
In the present section, we introduce some key concepts from the field of
software testing, as well as their applications to grammar testing.
% \subsection{Purpose of testing}

% Ammann and Offutt \todo{cite \url{http://assets.cambridge.org/97811071/72012/excerpt/9781107172012_excerpt.pdf}}
% define different levels of enlightment for software engineers.

% \begin{quote}
% Level 0:  There is no difference between testing and debugging. \\
% Level 1: The purpose of testing is to show correctness. \\
% Level 2: The purpose of testing is to show that the software does not work. \\
% Level 3: The purpose of testing is not to prove anything specific, but to reduce the risk of using the software. \\
% Level 4: Testing is a mental discipline that helps all IT professionals develop higher-quality software.
% \end{quote}

% \subsection{Testing}

% More quotes from the Intro to Software testing book:

% \begin{quote}
% Testing: Evaluating software by observing its execution \\
% Test Failure: Execution of a test that results in a software failure \\
% Debugging: The process of finding a fault given a failure \\
% \\
% Not all inputs will “trigger” a fault into causing a failure
% \\
% \\
% Reachability :The location or locations in the program that contain the fault must be reached \\
% Infection :The state of the program must be incorrect \\
% Propagation :The infected state must cause some output or  final state of the program to be incorrect \\
% Reveal :The tester must observe part of the incorrect portion of the program state \\
% \end{quote}

% http://twiki.di.uminho.pt/twiki/pub/Research/CROSS/Publications/techReport-ATGsurvey.pdf



\paragraph{Unit testing}

Unit tests are particular, concrete test cases: assume we want to test the
addition function (+), we could write some facts we know, such as
``1+2 should be 3''. In the context of grammars and natural language,
we could assert translation equivalence between a pair of strings,
e.g. ``\emph{la casa grande}'' $\Leftrightarrow$ ``\emph{the big house}'',
or between a tree and its linearisation, e.g. ``\t{DetCN the\_Det
  (AdjCN  big\_A house\_N)} $\Leftrightarrow$ \emph{the big house}''.
Whenever a program is changed or updated, the collection of unit tests
are run again, to make sure that the change has not broken something
that previously worked.

\paragraph{Property-based testing}

The weakness of unit testing is that it only tests concrete values
that the developer has thought of adding. Another approach is to use
property-based testing: the developer defines abstract properties that
should hold for all test cases, and uses random generation to supply
values. If we want to test the (+) function again, we could write a
property that says ``for all integers $x$ and $y$, $x+y$ should be
equal to $y+x$''.  A grammar-related property could be, for instance,
``a linearisation of a tree must contain strings from all of its
subtrees''.  We formulate these properties, and generate a large
amount of data---pairs of integers in the first case, syntax trees in
the second---and assert that the property holds for all of them.

% Of course, the single property of commutativity is not enough to
% define that (+) works as intended: if that were the only test, it
% would happily pass any function with similar property,
% e.g. multiplication, or some function $f(x, y) = 0$ for any $x$ and
% $y$. Likewise, some components in a syntax tree may not contribute
% with a string, but only a parameter to choose the right inflection
% form from another subtree.
%  To remedy this, we would need to test other
% properties of addition, or mix in some unit tests with specific
% values.


\paragraph{Deriving test cases}

Unit tests, as well as properties, can be written by a human or
derived automatically from some representation of the program. The
sources of tests can range from informal descriptions, such as
specifications or user stories, to individual statements in the source
code. Alternatively, tests can be generated from an abstract model of
the program, such as a UML diagram.

In the context of grammar testing, the specification is the whole
natural language that the grammar approximates---hardly a formal and
agreed-upon document. Assuming no access to the computational grammar
itself (generally called \emph{black-box testing}), we can treat traditional
grammar books or speaker intuition as an inspiration for properties
and unit tests. For example, we can test a feature such as ``pronouns
must be in an accusative case after a preposition'' by generating
example sentences where pronouns occur after a preposition, and
reading through them to see if the rule holds.

If we have access to the grammar while designing tests
(\emph{white-box testing}), we can take advantage of the
structure---to start with, only test those features that are
implemented in the grammar. For example, if we know that word order is
a parameter with 2 values, direct statement and question,
then we need to create 2 tests, e.g. ``I am old'' and ``am I
old''. If the parameter had 3 values, say third for indirect question,
then we need a third test as well, e.g. ``I don't know if I am old''.

% We can name
% three levels of abstraction where to generate tests. \emph{Black-box
%   testing} relies on external descriptions of the software, without
% access to the source code. A tester would not care how the program
% produces the intended output, just that its output is correct. In
% contrast, \emph{white-box testing} has access to the source code, and
% can exploit concrete details of the program, such as individual
% conditions and statements. Finally, \emph{model-based testing} derives
% tests from some model of a program, such as a UML diagram.


\paragraph{Coverage criteria}

Beizer \cite{beizer2003software} describes testing as a
simple task: ``all a tester needs to do is find a graph and cover
it''. The flow of an imperative program can be modelled as a graph
with start and end points; multiple branches at conditional statements
and back edges at loops. Take a simple program that takes a number and
outputs ``even'' for even numbers and ``odd'' for odd numbers; it has
two branches, and to test it exhaustively, one needs to supply two
inputs; one even and one odd number.

Simulating the run of the program with all feasible paths is called
\emph{symbolic evaluation}, and a constraint solver is often used to
find out where different inputs lead into. It is often not feasible to
simulate all paths for a large and complex program; instead, several
heuristics have been created for increasing code coverage.

Symbolic evaluation works well for analysing (ordered) \onlycg{} grammars, at
least up to an input space of tens of thousands of different
morphological analyses. The range of operations is fairly limited, and
the program flow is straightforward: execute rule 1, then execute
rule 2, and so on.

For \gf{} grammars, the notion of code coverage is based on individual
grammatical functions, rather than a program flow. We want to test
linearisation, not parsing, and thus our ``inputs'' are just syntax
trees.  We need to test all syntactic functions (e.g. putting an
adjective and a noun together), with all the words that make a
difference in the output (some adjectives come before the noun, others
come after). Thus, even if the grammar generates an infinite amount of
sentences, we still have only a finite set of constructions and
grammatical functions to cover.


\def\ant{\text{\em ant}}
\def\bat{\text{\em bat}}
\def\cat{\text{\em cat}}
\def\dog{\text{\em dog}}
\def\emu{\text{\em emu}}
\def\fox{\text{\em fox}}


\section{Boolean satisfiability (SAT)}
\label{sec:SAT-intro}

Imagine you are in a pet shop with a selection of animals: \ant, {\em bat, cat, dog, emu} and \fox.

These animals are very particular about each others' company. The dog has no teeth and needs the cat to chew its food. The cat, in turn, wants to live with its best friend bat. 
But the bat is very aggressive towards all herbivores, and the emu is afraid of anything lighter than 2 kilograms. The ant hates all four-legged creatures, and the fox can only handle one flatmate with wings. 

You need to decide on a subset of pets to buy---you love all animals, but due to their restrictions, you cannot have them all. You start calculating in your head: ``If I take the ant, I cannot take cat, dog, nor fox. How about I take the dog, then I must take the cat and the bat as well.''
After some time, you decide on bat, cat, dog and fox, leaving the ant and the emu in the pet shop.

\paragraph{Definition}

This everyday situation is an example of \emph{Boolean satisfiability (SAT)} problem.
The animals translate into \emph{variables}, %$\{ant, bat, cat, dog, emu, fox\}$.
and the cohabiting restrictions of each animal translate into \emph{clauses},
such that ``dog wants to live with cat'' becomes an implication $\dog \Rightarrow \cat$.
Under the hood, all of these implications are translated into even simpler constructs: lists of disjunctions.
For instance, ``dog wants to live with cat'' as a disjunction is %expressed as 
$\neg\dog \vee \cat{}$, which means ``I don't buy the dog or I buy the cat''.
%More complex implications need to be translated into two or more disjunctions.
The representation as disjunctions is easier to handle algorithmically; however,
for the rest of this thesis, we show our examples as implications, because they are easier to understand.
The variables and the clauses are shown in Figure~\ref{fig:animalCohabitingProblem}.


% Ultimately, the SAT-solver gets one large formula of disjunctions

% These clauses are disjunctions of \emph{literals}: either a variable, such as \cat, or its negation, $\neg\cat$.
% A positive literal \cat{} means that the animal comes with you, 
% and conversely, $\neg\cat{}$ means it is left in the shop. 


The objective is to find a \emph{model}: each variable is assigned a Boolean value, such that the conjunction of all clauses evaluates into true. A program called \emph{SAT-solver} takes the set of variables and clauses, and performs a search, like the mental calculations of the animal-lover in the shop.
We can see that the assignment $\{\ant=0, \bat=1, \cat=1, \dog=1, \emu=0, \fox=1\}$
satisfies the animals' wishes.
Another possible assignment would be $\{\ant=0, \bat=0, \cat=0, \dog=0, \emu=1, \fox=1\}$: you only choose the emu and the fox. 
Some problems have a single solution, some problems have multiple solutions, and some are unsatisfiable, i.e. no combination of assignments can make the formula true.

 \begin{figure}[t]
  % $$\begin{array}{r | r @{~} l | l @{\quad \wedge \quad } l @{\quad \wedge \quad} l }
  %  \textbf{Animal}
  %               & \multicolumn{2}{l}{\textbf{Constraint}} 
  %                                       & \multicolumn{3}{l}{\textbf{Constraint in conjunctive normal form}} \\ \hline

  %   \ant        & \ant &\Rightarrow \neg{}\cat \wedge \neg{}\dog \wedge \neg{}\fox 
  %   								    & \neg{}\ant \vee \neg{}\cat 
  %                                       & \neg{}\ant \vee \neg{}\dog 
  %                                       & \neg{}\ant \vee \neg{}\fox \\

  %  \bat         & \bat &\Rightarrow \neg{}\ant \wedge \neg{}\emu
  %  								        & \neg{}\bat \vee \neg{}\ant 
  %                                       & \multicolumn{2}{l}{\neg{}\bat \vee \neg{}\emu} \\
  %  \cat         & \cat &\Rightarrow \bat & \multicolumn{3}{l}{\neg{}\cat \vee \bat} \\
  %  \dog         & \dog &\Rightarrow \cat & \multicolumn{3}{l}{\neg{}\dog \vee \cat} \\
  %  \emu         & \emu &\Rightarrow \neg{}\ant \wedge \neg{}\bat
  %  										& \neg{}\emu \vee \neg{}\ant 
  %                                       & \multicolumn{2}{l}{\neg{}\emu \vee \neg{}\bat} \\
  %  \fox         & \fox &\Rightarrow \neg (\bat \wedge \emu) 
  %  									    & foo 
  %  									    & bar
  %  									    & baz \\
  %
  % \end{array}$$ \\




    $$\begin{array}{r | r @{~} l | l }
   \textbf{Variable}
                %& \multicolumn{2}{l}{\textbf{Constraint}} 
                & & \textbf{~~~Constraint} 
                                         & \textbf{Explanation} \\ \hline

                & \ant & \vee \ \bat \vee \cat \vee \dog \vee \fox & \text{``You want to buy at least one pet.''} \\
    \ant        & \ant &\Rightarrow \neg{}\cat \wedge \neg{}\dog \wedge \neg{}\fox & \text{``Ant does not like four-legged animals.''} \\
   \bat         & \bat &\Rightarrow \neg{}\ant \wedge \neg{}\emu
                          & \text{``Bat does not like herbivores.''} \\ 
   \cat         & \cat &\Rightarrow \bat & \text{``Cat wants to live with bat.''} \\
   \dog         & \dog &\Rightarrow \cat & \text{``Dog wants to live with cat.''} \\
   \emu         & \emu &\Rightarrow \neg{}\ant \wedge \neg{}\bat
                                         &  \text{``Emu does not like small animals.''} \\
   \fox         & \fox &\Rightarrow \neg (\bat \wedge \emu) 
                                         & \text{``Fox cannot live with two winged animals.''} 
  \end{array}$$ \\

  \caption{Animals' cohabiting constraints translated into a SAT-problem.}
  \label{fig:animalCohabitingProblem}
\end{figure}


% I actually like the CNF representation: it makes the problem look more low-level and simple. 
% I hope it would make the point "look, we reduce a complex problem into a simple format."

\paragraph{History and applications}

SAT-solving as a research area dates back to 1970s.
Throughout its history, it has been of interest for both theoretical and practical purposes. 
SAT is a well-known example of an \emph{NP-complete} (Nondeterministic Polynomial time) problem \cite{cook1971complexity}: for all such problems, a potential solution can be \emph{verified} in polynomial time, but there is no known algorithm that would \emph{find} such a solution, in general case, in sub-exponential time. 
This equivalence means that we can express any NP-complete problem as a SAT-instance, and use a SAT-solver to solve it.
The class includes problems which are much harder than the animal example;
nevertheless, all of them can be reduced into the same representation, just like $\neg\bat \vee \neg\emu$.


% think of tasks such as job scheduling, where jobs are scheduled to different actors with minimal overhead---$x$ must be completed before $y$ can start, and $z$ must run constantly, but not all actors can perform all jobs. 
% %This kind of optimisation is going on constantly in our computers, cars.
% Nevertheless, all these problems can be reduced into the same representation as the animals in the shop: 
% the massive effort of an operating system---or an autopilot---can be expressed as a set of simple disjunctions, just like $\neg\bat \vee \neg\emu$.

%Since the 2000s, SAT-solving is widely used in practical applications \cite{marques_silva2010}. 
The first decades of SAT-research was concentrated on the theoretical side, with little practical applications.
But things changed in the 90s: there was a breakthrough in the SAT-solving techniques, 
which allowed for scaling up and finding new use cases. As a result, modern SAT-solvers can deal with problems that have hundreds of thousands of variables and millions of clauses \cite{marques_silva2010}.

What was behind these advances? SAT, as a general problem, remains NP-complete: 
it is true that there are still SAT-problems that cannot be solved in sub-exponential time.
However, there is a difference between a general case, where the algorithm must be prepared for any input, and an ``easy case'', where we can expect some helpful properties from the input.
Think of a sorting algorithm: in the general case, it is given truly random lists, and in the ``easy case'', it mostly gets lists with some kind of structure, such as half sorted, reversed, or containing bounded values. The general time complexity of sorting is still $O(n\ log\, n)$, but arguably, the easier cases can be expected to behave in linear time, and we can even design heuristic sorting algorithms that exploit those properties.

Analogously, the 90s breakthrough was due to the discovery of right kind of heuristics.
%Turns out that many typical real-life scenarios can be formulated as a SAT-problem of the ``easy kind''.
% when formulated as SAT-problem, can be solved efficiently with proper heuristics. 
Much of the SAT-research in the last two decades has been devoted to optimising the solving algorithms, and finding more efficient methods of encoding various real-life problems into SAT.
This development has led to an increasing amount of use cases since the early 2000s \cite{claessen2009satpractice}.
One of the biggest success stories for a SAT-application is model checking \cite{sheeran1998modelchecking,biere1999modelchecking,bradley2011modelchecking}, used in software and hardware verification. Furthermore, SAT has been used in domains such as computational biology \cite{claessen2013compbioSAT} and AI planning \cite{selman_kautz92aiplanning}, just to pick a few examples.
In summary, formulating a decision problem in SAT is an attractive approach: instead of developing search heuristics for each problem independently, one can transform the problem into a SAT-instance and exploit decades of research into SAT-solving. 

% Instead of implementing our own search, we can exploit decades of research that has been devoted to SAT-solving: optimising the search and finding the best heuristics, as well as innovative ways of formulating SAT-problems.



\section{Summary}

In this section, we have presented the theoretical background used in this thesis.  We have introduced Constraint Grammar and Grammatical Framework as examples of computational grammars.  On the software testing side, we have presented key concepts such as unit testing and Boolean satisfiability. In the following chapters, we will connect the two branches. Firstly, we encode CG grammars as a SAT-problem, which allows us to apply symbolic evaluation and find potential conflicts between rules.  Secondly, we use methods for creating minimal and representative test data, in order to find a set of trees that test a given GF grammar in an optimal way.
