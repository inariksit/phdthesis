
\section{Constraint Grammar}
\label{sec:cg-intro}

Constraint Grammar (CG) is a formalism for 
disambiguating morphologically analysed text. 
It was first introduced by Fred Karlsson  
\cite{karlsson1990cgp,karlsson1995constraint}, and has been used for many tasks in
computational linguistics, such as part-of-speech tagging, surface syntax and
machine translation \cite{bick2011}.
CG-based taggers are reported to achieve F-scores of over 99 \% for morphological disambiguation, 
and around 95-97 \% for syntactic analysis \cite{bick2000palavras,bick2003hybridCG_PSG,bick2006spanish}.
CG disambiguates morphologically analysed input by using
constraint rules which can select or remove a potential analysis (called \emph{reading})
for a target word, depending on the context words around it. 
Together these rules disambiguate the whole text.


In the example below, we show an initially ambiguous sentence ``the bear
sleeps''. 
It contains three word forms, such as \t{"<bear>"}, each followed by its \emph{readings}.
A reading contains one lemma, such as \t{"bear"}, and a list of morphological tags, such as \t{noun sg}.
%Additional lemmas within one word form, such as clitic pronouns, are represented as \emph{subreadings}; each indented one more tab.
A word form together with its readings is called a \emph{cohort}. A cohort is ambiguous, if it contains more than one reading.

\begin{figure}[h]
\centering
\ttfamily
\begin{tabular}{p{0.6cm} l  p{0.6cm} l}
"<the>"  &                & "<sleeps>"        \\
    & "the" det def       &     & "sleep" noun pl \\
"<bear>" &                &     & "sleep" verb pres p3 sg \\
    & "bear" noun sg      & "<.>"                   \\
    & "bear" verb pres    &     & "." sent          \\
    & "bear" verb inf \\
\end{tabular}
\label{fig:theBearSleeps}
\caption{Ambiguous sentence {\em the bear sleeps.}}
\end{figure}


\noindent We can disambiguate this sentence with two rules:

\begin{enumerate}
\def\labelenumi{\arabic{enumi}.}
\itemsep1pt\parskip0pt\parsep0pt
\item \texttt{REMOVE verb IF (-1 det)}
  `Remove verb after determiner'
\item  \texttt{REMOVE noun IF (-1 noun)}
  `Remove noun after noun'
\end{enumerate}

\noindent Rule 1 matches the word \emph{bear}: it is tagged as verb and is
preceded by a determiner. The rule removes both verb readings from
\emph{bear}, leaving it with an unambiguous analysis \texttt{noun sg}.
Rule 2 is applied to the word \emph{sleeps}, and it removes the noun
reading. The finished analysis is shown below:

\begin{itemize}
\item[] 
\begin{verbatim}
"<the>"
        "the" det def
"<bear>"
        "bear" noun sg
"<sleeps>"
        "sleep" verb pres p3 sg
\end{verbatim}
\end{itemize}

It is also possible to add syntactic tags and dependency structure within CG \cite{vislcg3,bick2015}.
However, for the remainder of this introduction, we will illustrate the examples with the 
most basic operations, that is, disambiguating morphological tags.
The syntactic disambiguation and dependency features are not fundamentally
different from morphological disambiguation: the rules describe an \emph{operation}
performed on a \emph{target}, conditional on a \emph{context}.

\subsection{Related work}

CG is one in the family of shallow and reductionist grammar formalisms. Disambiguation using constraint rules dates back to 1960s and 1970s---the closest system to modern CG was Taggit \cite{taggit}, which was used to tag the Brown Corpus.
Karlsson \cite{karlsson1995constraint} lists various approaches to the disambiguation 
problem, including manual intervention, statistical optimisation, unification and Lambek calculus. 
For disambiguation rules based on local constraints, Karlsson mentions \cite{hindle1989disamrules,herz1991local}.

CG itself was introduced in 1990. Around the same time, a related formalism was proposed: 
finite-state parsing and disambiguation system using constraint rules \cite{koskenniemi90}, which was later named Finite-State Intersection Grammar (FSIG) in \cite{piitulainen1995}. 
Like CG, a FSIG grammar contains a set of rules which remove impossible readings based on contextual tests, 
in a parallel manner: a sentence must satisfy all individual rules in a given FSIG. 
Due to these similarities, the name Parallel Constraint Grammar was also suggested \cite{koskenniemi97}.
However, FSIG rules usually target syntactic phenomena, and the rules allow for more expressive contextual tests than CG. 
In this thesis, we use the name FSIG to refer to the framework that is aimed at producing a full syntactic analysis with the more expressive rules, 
and PCG to describe any CG-variant which happens to be parallel.
Thus, we will call implementations such as \cite{voutilainen1994designing} an \emph{FSIG grammar} and a program that parses it an \emph{FSIG parser}.
Conversely, the CG parser in \cite{listenmaa_claessen2015} 
and the ``CG-like'' system in \cite{lager98} are instances of PCG.
%We return to the comparison with FSIG in Sections~\ref{sec:ordering}~and~\ref{sec:expressivity}.

Brill tagging \cite{brill1995} is based on transformation rules: the starting point of 
an analysis is just one tag, the most common one, and subsequent rule applications transform 
one tag into another, based on contextual tests. 
Like CG, Brill tagging is known to be efficient and accurate. The contextual tests are very similar; \cite{lager01transformation} 
has automatically learned both Brill rules and CG rules, using the same system.

Similar ideas to CG have been also explored in other frameworks, such as finite-state automata \cite{gross1997local,grana2003fst},
%, such as Local grammars \cite{gross1997local} and finite-state contextual rules \cite{grana2003fst},
logic programming \cite{oflazer97votingconstraints,lager98}, 
constraint satisfaction \cite{padro1996csp}, 
and dependency syntax \cite{tapanainen97fdg}. 
 In addition, there are a number of reimplementations of CG using finite-state methods \cite{yli-jyra2011cg_engine,hulden2011cg_engine,peltonen2011}. 

\subsection{Properties of Constraint Grammar}\label{sec:properties}

Karlsson \cite{karlsson1995constraint} lists 24 design principles and describes
related work at the time of writing.
Here we summarise a set of main features, and relate CG to the developments in grammar formalism since the initial description.

CG is a \emph{reductionistic} system: the analysis starts from a list of alternatives,
and removes those which are impossible or improbable.
CG is designed primarily for analysis, not generation; its task is 
to give correct analyses to the words in given sentences,
not to describe a language as a collection of ``all and only the grammatical sentences''.

The syntax is decidedly \emph{shallow}: the rules do not aim to
describe all aspects of an abstract phenomenon such as noun phrase; 
rather, each rule describes bits and pieces with concrete conditions.
The rules are self-contained and mutually independent---this makes it 
easy to add exceptions, and exceptions to exceptions, without 
changing the more general rules.
% The rules are self-contained and independent.
% On the one hand, this provides no guarantee that a grammar is internally consistent.
% On the other hand, these features provide flexibility that is hard to mimic by a deeper formalism.
% As we have seen in the previous sections, rules can target individual words
% or other properties that are not generalisable to a whole word class,
% such as verbs that express cognitive processes.
% Introducing a subset of verbs, even if they are used only in one rule,
% is very cheap and does not create a complicated taxonomy of different verb types.

% Most importantly, the independence of rules makes CG highly robust.
% If one of the words is unknown or misspelt, a generative grammar would fail to produce any analysis. 
% CG would, at worst, just leave that part ambiguous, and do as good a job it can elsewhere in the sentence.


There are different takes on how \emph{deterministic} the rules are.
The current state-of-the-art CG parser VISL CG-3 executes the rules strictly 
based on the order of appearance, but there are other implementations which 
apply their own heuristics, or remove the ordering completely, 
applying the rules in parallel. 
---furthermore, there are several implementations of CG with different kind of application orders,
such as ``apply in the order introduced in the grammar file'' or ``apply in order of longer matching context''.
A particular rule set may be written with one application order in mind, but another party may 
run the grammar with another implementation---if there are any conflicting rule pairs, then the behaviour of the grammar is different.
For that reason, we decided to apply software testing and verification methods to CG grammars.





\section{Grammatical Framework}
\label{sec:gf-intro}

Grammatical Framework (GF) \cite{ranta2011gfbook} 
is a framework for building multilingual grammar applications. Its main
components are a functional programming language for writing grammars
and a resource library that contains the linguistic details of many
natural languages.
A GF program consists of an \emph{abstract syntax} (a set of functions
and their categories) and a set of one or more
\emph{concrete syntaxes} which describe how the abstract
functions and categories are linearized (turned into surface strings) in each
respective concrete language. The resulting grammar
describes a mapping between concrete language strings and
their corresponding abstract trees (structures of function names).
This mapping is bidirectional---strings can be \emph{parsed} to
trees, and trees \emph{linearised} to strings.
As an abstract syntax can have multiple corresponding concrete syntaxes,
the respective languages can be automatically \emph{translated} from one to the other by
first parsing a string into a tree and then linearising the obtained tree
into a new string.

\subsection{Theoretical background/related work}

Type theory and logical frameworks; Montague grammar; Categorial grammar.

Abstract and concrete syntax: compiler construction \& tectogrammatical/phenogrammatical \todo{cite (Curry 1961)}. Multi-source multi-target compiler is multilingual by design, but tecto/pheno stuff wasn't multilingual before GF.

\cite{ranta2011gfbook} mentions \todo{cite relevant stuff}


\subsection{Abstract syntax}

\subsection{Concrete syntax}
