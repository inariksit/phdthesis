\section{Constraint Grammar}
\label{sec:cg-intro}

\todo{cut down the intro in my lic.}

\section{Grammatical Framework}
\label{sec:gf-intro}

Grammatical Framework (GF) \cite{ranta2011gfbook} 
is a framework for building multilingual grammar applications. Its main
components are a functional programming language for writing grammars
and a resource library that contains the linguistic details of many
natural languages.
A GF program consists of an \emph{abstract syntax} (a set of functions
and their categories) and a set of one or more
\emph{concrete syntaxes} which describe how the abstract
functions and categories are linearized (turned into surface strings) in each
respective concrete language. The resulting grammar
describes a mapping between concrete language strings and
their corresponding abstract trees (structures of function names).
This mapping is bidirectional---strings can be \emph{parsed} to
trees, and trees \emph{linearized} to strings.
As an abstract syntax can have multiple corresponding concrete syntaxes,
the respective languages can be automatically \emph{translated} from one to the other by
first parsing a string into a tree and then linearizing the obtained tree
into a new string.
