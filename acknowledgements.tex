\chapter*{Acknowledgements}\label{chp:acknowledgements}



% % It's been fun and rewarding and tough and all kinds of adjectives.  I
% % don't think one ever feels finished, but for practical purposes, 5
% % years is a fairly good checkpoint.

% % The first round of thanks goes to my supervisors Koen Claessen and
% % Aarne Ranta.


I am grateful to my supervisor Koen Claessen for his support and encouragement.
I cherish our hands-on collaboration, which often involved sitting in the same
room writing code together. You have been there for all aspects of a PhD,
both inspiration and perspiration!
% and
% Who knew I would find not just one but two answers to life, universe
% and everything---I'm of course talking about SAT and fixpoint
% computation.
My co-supervisor Aarne Ranta is an equally important figure in my PhD
journey, and a major reason I decided to pursue a PhD in the first
place. The reader of this thesis may thank Aarne for making me rewrite
certain sections over and over again, until they started making sense!
%
Furthermore, I want to thank my opponent Fred Karlsson, and Torbjörn
Lager, Gordon Pace and Laurette Pretorius for agreeing to be in my
grading committee, as well as Graham Kemp for being my examiner.

Looking back, I have an enormous gratitude the whole GF community.
It all started back in 2010 when I was a master's student in Helsinki
and joined a research project led by Lauri Carlson. Lauri Alanko was
most helpful office mate when I was learning GF, Kaarel Kaljurand was
my first co-author and the Estonian resource grammar we wrote was my
first large-scale GF project.  As important as it was to learn from
others during my first steps, becoming a teacher myself has been even
more enlightening. I am happy to have met and guided newer GF
enthusiasts, especially Bruno Cuconato and Kristian Kankainen---I may
have learned more from your questions than you from my answers!

During my PhD studies, I've had the chance to collaborate with several
people outside Gothenburg and my research group. I want to thank
Eckhard Bick, Tino Didriksen and Francis Tyers for introducing me to
CG and keeping up with my questions and weird ideas.  I am grateful to
Jose Mari Arriola and Maxux Aranzabe for hosting my research visit in
the Basque country and giving me a chance to work with the coolest
language I've seen so far---eskerrik asko!

In addition to my roles as a student, teacher and colleague, I've enjoyed
more unstructured exploration among peers, and pursuing interesting side tracks.
Wen Kokke deserves special thanks for making such an unlikely collaboration
happen; as well as for reading a number of nonsensical test sentences in Dutch,
and for everything else!

On a more day-to-day basis, I want to thank all my colleagues at the
department. My office mates Herb and Prasanth have shared with me joys
and frustrations, and helped me to decipher supervisor comments. In
addition, I thank the whole language technology group: John, Krasimir,
Peter, Ramona, Thomas \todo{formulate this somehow nicely}.

Outside my office, I want to thank Anders, Dan, Daniel, Elena, Irene,
Gabriel, Guilhem, Simon H., Simon R. and Víctor for all the fun things
during the 5 years: interesting lunch discussions, fermentation
parties, hairdyeing parties, climbing, board games, forest excursions,
playing music together, sharing a houseboat---just to name a few
things. (Also it's handy to have a stack of your old theses in my
office to use as an inspiration for writing acknowledgements!)

Sometimes it's also fun to meet people outside research! I want to
thank the wonderful people in Kulturkrock and Chalmers sångkör for
making me feel at home in Sweden, not just in the small bubble of my
research group. Ett extra tack till Anka och Jonatan för att ni rättat
min svenska!

All the factors I've mentioned previously have been important in finishing
my PhD. Listing all the factors that enabled me to start a PhD would take
more space than the actual monograph, so let me be brief and thank my parents,
firstly, for making me exist, and secondly, for raising me to believe in myself,
be willing to take challenges, and never stop learning.

\vfill\noindent
This work has been carried out within the REMU project — Reliable Multilingual Digital Communication: Methods and Applications.
The project was funded by the Swedish Research Council (\emph{Vetenskapsrådet}) under grant number 2012-5746.
