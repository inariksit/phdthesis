\chapter*{Acknowledgements}\label{chp:acknowledgements}



% % It's been fun and rewarding and tough and all kinds of adjectives.  I
% % don't think one ever feels finished, but for practical purposes, 5
% % years is a fairly good checkpoint.

% % The first round of thanks goes to my supervisors Koen Claessen and
% % Aarne Ranta.  


I am grateful to my supervisor Koen Claessen for his support and encouragement.
% and
Who knew I would find not just one but two answers to life, universe
and everything---I'm of course talking about SAT and fixpoint
computation.
% In the beginning, we both had our separate bubbles of knowledge, and
% this thesis was born out of combining them. 
My co-supervisor Aarne Ranta is an equally important figure in my PhD
journey, and a major reason I decided to pursue a PhD in the first
place. 
You two have been expert guides outside my comfort zone---I have felt
that my language background is truly appreciated, meanwhile I have
been learning more computer science

Furthermore, I want to thank my opponent Fred Karlsson, Torbjörn
Lager, Gordon Pace and Laurette Pretorius for agreeing to be in my
grading committee, as well as Graham Kemp for being my examiner.


I want to thank the whole GF community.
It all started back in 2010 when I was a master's student and joined a
research project in Helsinki, led by Lauri Carlson. Lauri Alanko was
most helpful office mate when I was learning GF, Kaarel Kaljurand was
my first co-author and the Estonian resource grammar we wrote was my
first large-scale GF project.  As important as it was to learn from
others during my first steps, becoming a teacher myself has been even
more enlightening. I am happy to have met and guided newer GF
enthusiasts, especially Bruno Cuconato and Kristian Kankainen---I may
have learned more from your questions than you from my answers!

During my PhD studies, I've had the chance to collaborate with several
people outside Gothenburg and my research group. I want to thank
Eckhard Bick, Tino Didriksen and Francis Tyers for introducing me to
CG and keeping up with my questions and weird ideas.  I am grateful to
Jose Mari Arriola and Maxux Aranzabe for hosting my research visit in
the Basque country and giving me a chance to work with the coolest
language I've seen so far---eskerrik asko!

Wen Kokke deserves a number of special thanks for being an awesome
co-author (despite being part of a whole another grammar community and
doing a PhD in a completely non-grammar-related area), and for everything else!

On a more day-to-day basis, I want to thank all my colleagues at the
department. My office mates Herb and Prasanth have shared with me joys
and frustrations (and dirty \LaTeX{} hacks), and helped me to decipher
supervisor comments. Outside my office, I want to thank Anders, Dan,
Daniel, Elena, Irene, Gabriel, Guilhem, Simon H., Simon R. and Víctor
for all the fun things during the 5 years: interesting lunch
discussions, fermentation parties, hairdyeing parties, climbing, board
games, forest excursions, playing music together, sharing a
houseboat---just to name a few things. (Also it's handy to have a
stack of your old theses in my office to use as an inspiration for
writing acknowledgements!)

Sometimes it's also fun to meet people outside research! I want to
thank the wonderful people in Kulturkrock and Chalmers sångkör for
making me feel at home in Sweden, not just in the small bubble of my
research group. Ett extra tack till Anka och Jonatan för att ni rättat
min svenska!

I could go on for quite a while, but since this thesis is all about
grammars, let me express my gratitude in a way that only a grammarian can.

\begin{verbatim}
abstract ThankYou = {
  flags startcat = Greeting ;
  cat 
    Greeting ; Recipient ;
  fun
    Thanks : Recipient -> Greeting ;
    Koen, Aarne, Colleagues, Friends, Family : Recipient ;
}
\end{verbatim}

%% \todo{Write proper acknowledgements}

\vfill\noindent
This work has been carried out within the REMU project — Reliable Multilingual Digital Communication: Methods and Applications.
The project is funded by the Swedish Research Council (\emph{Vetenskapsrådet}) under grant number 2012-5746.
