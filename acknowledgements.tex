\chapter*{Acknowledgements}\label{chp:acknowledgements}

It's been fun and rewarding and tough and all kinds of adjectives.  I
don't think one ever feels finished, but for practical purposes, 5
years is a fairly good checkpoint.

The first round of thanks goes to my supervisors Koen Claessen and
Aarne Ranta. Who knew I would find not just one but two answers to
life, universe and everything---I'm of course talking about SAT and
fixpoint computation. You have encouraged me to leave my
comfort zone, and to become a better researcher and person.
Furthermore, I want to thank \todo{My eventual opponent/grading
  committee} for becoming my opponent/ grading
committee, %, and providing feedback on the draft (if they do!)
and Graham Kemp for being my examiner.


During my PhD, I've had the chance to collaborate with several people
outside my research group. I want to thank Eckhard Bick, Tino
Didriksen and Francis Tyers for introducing me to CG and keeping up
with my questions and weird ideas.  I am grateful to Jose Mari Arriola
and Maxux Aranzabe for hosting my research visit in the Basque country
and giving me a chance to work with the coolest language I've seen so
far---eskerrik asko!

I want to also thank the whole GF community, which I was already a
part of before starting my PhD. It all started back in 2010 when I was
a master's student and joined a research project in Helsinki, led by
Lauri Carlson. Lauri Alanko was very helpful (and fun!)  office mate
and helped me to learn GF, Kaarel Kaljurand was my first co-author and
the Estonian resource grammar we wrote was my first major GF project.
% This list could go on for much further, but let me just conclude
% with two special mentions: first one to Kristian Kankainen, for
% organising an improvised summer school at his place (where this
% thesis is getting its final form), just because the official summer
% schools are only every second year!  Another one goes to Bruno
% Cuconato, for being the first one outside  (from the other side of the
% world!) to use the tools I created.

Wen Kokke deserves a number of special thanks for being an awesome
co-author (despite being part of a whole another grammar community and
doing a PhD in a completely non-grammar-related area), for suggesting
the title of this thesis, and for everything else!

On a more day-to-day basis, I want to thank all my colleagues at the
department. My office mates Herb and Prasanth have shared with me joys
and frustrations (and dirty \LaTeX{} hacks), and helped me to decipher
supervisor comments. Outside my office, I want to thank Anders,
Bassel, Dan, Daniel, Elena, Irene, Gabriel, Guilhem, Simon H., Simon
R. and Víctor for all the fun things during the 5 years: interesting
lunch discussions, fermentation parties, hairdyeing parties, climbing,
board games, forest excursions, playing music together, sharing a
houseboat---just to name a few things. (Also it's handy to have a
stack of your old theses in my office to use as an inspiration for
writing acknowledgements!)

% Det finns också liv utanför forskning! Ett stort tack till Kulturkrock
% och Chalmers sångkör för att balansera mitt liv med sång och
% dans. % och konstiga frisyrer.
% Dessutom är jag glad att ni har haft talamod att prata svenska med
% mig, även när jag inte kunde uttrycka mig så bra---extra tack till
% Jonatan och Anka för att rätta min svenska \todo{, bland annat dessa rader}!

% I could go on for quite a while, but since this thesis is all about
% grammars, let me express my gratitude in a way that only a grammarian can.

% \begin{verbatim}
% abstract ThankYou = {
%   flags startcat = Greeting ;
%   cat 
%     Greeting ; Recipient ;
%   fun
%     Thanks : Recipient -> Greeting ;
%     Koen, Aarne, Colleagues, Friends, Family : Recipient ;
% }
% \end{verbatim}


\vfill\noindent
This work has been carried out within the REMU project — Reliable Multilingual Digital Communication: Methods and Applications.
The project is funded by the Swedish Research Council (\emph{Vetenskapsrådet}) under grant number 2012-5746.
