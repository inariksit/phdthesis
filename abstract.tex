\chapter*{Abstract}\label{chp:abstract}

%% Abstract page

%% The third page must have a short reference (abstract).

%% The reference must:
%% Be written in English.
%% Contain the name of the dissertation, possible subtitle, name of the author, department and 'Chalmers University of Technology'.
%% If the thesis is written in Swedish this should also be stated.
%% Be concise and do not exceed more than approximately 250 words.
%% Provide a brief, easily understood overview of the essential content of the thesis (problems, methods, results).
%% Conclude with a maximum of ten key words of significance for computerised information systems.

%% The reference can also be printed on the back of the presentation
%% sheet, see below.

\todo{Write a proper abstract}

Testing grammars has one big difference from testing software: natural
language has no formal specification, so ultimately we must involve a
human oracle. However, we can automate many useful subtasks: detect
\emph{ambiguous constructions} and \emph{contradictory grammar rules},
as well as generate \emph{minimal and representative} set of examples
that cover all the constructions.  Think of the whole grammar as a
haystack, and we suspect there are a few needles---we cannot promise
automatic needle-removal, but instead we help the human oracle to
narrow down the search.
% \textbf{\lictitle}\\
% \textit{\licsubtitle}\\
% \textsc{\licauthor}\\
% \licdepartment\\
% \textsc{\licuniversity}\\



\bigskip
\noindent
\textbf{Keywords}: \emph{\phdkeywords}
