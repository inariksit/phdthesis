\chapter*{Abstract}\label{chp:abstract}

%% Abstract page

%% The third page must have a short reference (abstract).

%% The reference must:
%% Be written in English.
%% Contain the name of the dissertation, possible subtitle, name of the author, department and 'Chalmers University of Technology'.
%% If the thesis is written in Swedish this should also be stated.
%% Be concise and do not exceed more than approximately 250 words.
%% Provide a brief, easily understood overview of the essential content of the thesis (problems, methods, results).
%% Conclude with a maximum of ten key words of significance for computerised information systems.

%% The reference can also be printed on the back of the presentation
%% sheet, see below.
 

Grammar engineering has a lot in common with software engineering. 
Analogous to a program specification, we use descriptive grammar
books; in place of unit tests, we have gold standard corpora and test
cases for manual inspection. And just like any software, our grammars
still contain bugs: grammatical sentences that are rejected,
ungrammatical sentences that are parsed, or grammatical sentences that
get the wrong parse.

This thesis presents two contributions to the analysis and quality
control of computational grammars of natural languages. Firstly, we
present a method for finding \emph{contradictory grammar rules} in
Constraint Grammar, a robust and low-level formalism for
part-of-speech tagging and shallow parsing.  Secondly, we generate
\emph{minimal and representative test suites} of example sentences
that cover all grammatical constructions in Grammatical Framework, a
multilingual grammar formalism based on deep structural analysis.


% We use established techniques from software testing and formal
% methods and apply them to two different systems: Constraint Grammar,
% a robust and low-level formalism for part-of-speech tagging and
% shallow parsing, and Grammatical Framework, a multilingual grammar
% formalism based on deep structural analysis.  



\bigskip
\noindent
\textbf{Keywords}: \emph{\phdkeywords}
